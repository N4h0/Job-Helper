\begin{greenbox}
    \textbf{Nøkkelkvalifikasjonar}
    \setlength{\baselineskip}{3pt}

    \begin{itemize}
        \item utdanna barnehagelærer fra Høgskulen på Vestlandet (Bachelorgrad)
        \item erfaring med planlegging, gjennomføring, dokumentasjon og evaluering av det pedagogiske arbeidet
        \item evne til å leie og utvikle personalets potensiale
        \item sterkt engasjement for barnehagefaget og fokus på faglig oppdatering
        \item vektlegger kunnskap, samarbeid, initiativ, og refleksjon i arbeidet mitt
    \end{itemize}
\end{greenbox}

\makerubrichead{Utdanning}

\entry[08.2018 -- 06.2021]{\textbf{Bachelorgrad i barnehagelærer, Høgskulen på Vestlandet} \par
utdanninga har gitt meg spesielt brei kunnskap om barns utvikling, lek og læring, og det å skape gode omsorgs- og læringsmiljøer for barnehagebarn.}
\vspace{2pt}

\makerubrichead{Arbeidserfaring}

\entry[08.2021 -- 12.2021]{\textbf{Pedagogisk leder, Solheim barnehage} \par
Ansvar for planlegging, gjennomføring, dokumentering og evaluering av det pedagogiske arbeidet i min gruppe. Jobbet tett med resten av ledergruppa for å sikre god kvalitet og utvikling i barnehagen.}
\vspace{2pt}
\entry[08.2016 -- 06.2018]{\textbf{Assistent, Fjell barnehage, deltid} \par
Som assistent deltok jeg aktivt i det pedagogiske arbeidet, støttet barna i deres utvikling og lek, og bidro til et trygt og inkluderende miljø.}
\vspace{2pt}

\makerubrichead{Ferdigheiter}

\entry[Barns utvikling, lek og læring]{Brei teoretisk og praktisk forståing av barns utvikling, lek og læring.}
\vspace{2pt}
\entry[Planlegging, gjennomføring, dokumentasjon og evaluering]{Evne til å håndtere heile syklusen av det pedagogiske arbeidet.}
\vspace{2pt}
\entry[Endringsarbeid og utvikling]{Fokus på utvikling for å møte endringar og nye utfordringar i barnehagesektoren.}
\vspace{2pt}
\entry[Språk]{sterk munnleg og skriftleg kompetanse i nynorsk og engelsk}
\vspace{2pt}
