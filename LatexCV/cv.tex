%%%%%%%%%%%%%%%
% This CV example/template is based on my own
% CV which I (lamely attempted) to clean up, so that
% it's less of an eyesore and easier for others to use.
%
% LianTze Lim (liantze@gmail.com)
% 23 Oct, 2022

\documentclass[a4paper,skipsamekey,11pt,english]{curve}

% Uncomment to enable Chinese; needs XeLaTeX
% \usepackage{ctex}

% Default biblatex style used for the publication list is APA6. If you wish to use a different style or pass other options to biblatex you can change them here. 
\PassOptionsToPackage{style=ieee,sorting=ydnt,uniquename=init,defernumbers=true}{biblatex}

% Most commands and style definitions are in settings.sty.
\usepackage{settings}

% If you need to further customise your biblatex setup e.g. with \DeclareFieldFormat etc please add them here AFTER loading settings.sty. For example, to remove the default "[Online] Available:" prefix before URLs when using the IEEE style:
\DefineBibliographyStrings{english}{url={\textsc{url}}}

%% Only needed if you want a Publication List
\addbibresource{own-bib.bib}

%% Specify your last name(s) and first name(s) (as given in the .bib) to automatically bold your own name in the publications list. 
%% One caveat: You need to write \bibnamedelima where there's a space in your name for this to work properly; or write \bibnamedelimi if you use initials in the .bib
% \mynames{Lim/Lian\bibnamedelima Tze}

%% You can specify multiple names like this, especially if you have changed your name or if you need to highlight multiple authors. See items 6–9 in the example "Journal Articles" output.
\mynames{Lim/Lian\bibnamedelima Tze,
  Wong/Lian\bibnamedelima Tze,
  Lim/Tracy,
  Lim/L.\bibnamedelimi T.}
%% MAKE SURE THERE IS NO SPACE AFTER THE FINAL NAME IN YOUR \mynames LIST


% Change the fonts if you want
\ifxetexorluatex % If you're using XeLaTeX or LuaLaTeX
  \usepackage{fontspec} 
  %% You can use \setmainfont etc; I'm just using these font packages here because they provide OpenType fonts for use by XeLaTeX/LuaLaTeX anyway
  \usepackage[p,osf,swashQ]{cochineal}
  \usepackage[medium,bold]{cabin}
  \usepackage[varqu,varl,scale=0.9]{zi4}
  \usepackage{tabularx}
\else % If you're using pdfLaTeX or latex
  \usepackage[T1]{fontenc}
  \usepackage[p,osf,swashQ]{cochineal}
  \usepackage{cabin}
  \usepackage[varqu,varl,scale=0.9]{zi4}
  \usepackage[most]{tcolorbox}
  \usepackage{xcolor}
  \usepackage{tabularx}
\fi

\newlength{\entrylabelwidthcustom}
\setlength{\entrylabelwidthcustom}{3.8cm} % Tweak this if needed




\definecolor{darkerGreen}{HTML}{E6F2DA}
\definecolor{lighterGreen}{HTML}{B3CC80} % A much lighter shade of green

% Define the greenbox tcolorbox environment
\newtcolorbox{greenbox}{
  colback=lighterGreen!30, % Much less bright green
  colframe=darkerGreen, % Darker green for the frame
  boxrule=0.5mm,
  boxsep=5mm,
  arc=0mm, % Makes corners square
  left=10mm, % Extra space on the left side for text
  right=10mm, % Balance the space on the right side
  top=0mm, % Top padding
  bottom=0mm, % Bottom padding
  width=\textwidth, % Box width set to text width
  halign=left % Text aligned to the left
}

% Change the page margins if you want
% \geometry{left=1cm,right=1cm,top=1.5cm,bottom=1.5cm}

% Change the colours if you want
% \definecolor{SwishLineColour}{HTML}{00FFFF}
% \definecolor{MarkerColour}{HTML}{0000CC}

% Change the item prefix marker if you want
% \prefixmarker{$\diamond$}

%% Photo is only shown if "fullonly" is included
\includecomment{fullonly}
% \excludecomment{fullonly}

%%%%%%%%%%%%%%%%%%%%%%%%%%%%%%%%%%%%%%

\leftheader{%
  {\LARGE\bfseries\sffamily Johan Tryti}\\[8pt]

  % First line of contact fields
  \makefield{\faEnvelope[regular]}{\href{mailto:johantryti@gmail.com}{\texttt{johantryti@gmail.com}}}
  \hspace{1.5em}
  \makefield{\faLinkedin}{\href{https://www.linkedin.com/in/johan-tryti-a0a5a017a/}{\texttt{linkedin}}}
  \hspace{1.5em}
  % Second line of contact fields
  \makefield{\faGithub}{\href{https://github.com/N4h0}{\texttt{GitHub}}}\\[3pt]  % <-- Small vertical gap
  \makefield{\faPhone}{\texttt{+47 413 63 740}}
}

\rightheader{~}
\begin{fullonly}
\photo[r]{photo.jpg}
\photoscale{0.13}
\end{fullonly}

\title{Curriculum Vitae}



\begin{document}

\makeatletter
\renewcommand{\entry}[2][]{%
  \par\noindent%
  \parbox[t]{\linewidth}{%
    \makebox[\entrylabelwidthcustom][l]{\textbf{#1}}%
    \parbox[t]{\dimexpr\linewidth-\entrylabelwidthcustom\relax}{#2}%
  }%
  \par\vspace{4pt}%
}
\makeatother

\makeheaders[c]

\begin{greenbox}
    \textbf{Nøkkelkvalifikasjonar}
    \setlength{\baselineskip}{3pt}

    \begin{itemize}
        \item universitets-/høgskoleutdanning (bachelornivå) innenfor fagområdet kjemi
        \item gode skriftlige og muntlige ferdigheter i norsk og engelsk
        \item arbeidserfaring innenfor organisk kjemi og helse, miljø og sikkerhetsarbeid
        \item kompetanse innen bærekraft, ytre miljø, stråling og strålevern
        \item gode kommunikasjonsevner, løsningsorientert, selvgående
        \item strukturert, analytisk, systematisk
    \end{itemize}
\end{greenbox}

\makerubrichead{Utdanning}

\entry[08.2016 -- 06.2019]{\textbf{Bachelorgrad i kjemi, NMBU}}
\vspace{2pt}
\entry[08.2019 -- 09.2021]{\textbf{Mastergrad i radioøkologi, NMBU} \par
fokus på helse, miljø og sikkerhetsarbeid innen organisk kjemi og radioøkologi}
\vspace{2pt}

\makerubrichead{Arbeidserfaring}

\entry[08.2021 -- Present]{\textbf{Resepsjonist og instruktør, Mudo Gym Torshov, deltid}}
\vspace{2pt}
\entry[08.2020 -- 08.2021]{\textbf{Saksbehandler kjemikalier, Forsvarsmateriell} \par
Saksbehandling av kjemikalier, Ivaretakelse av materiellsikkerhet, Rådgivning innenfor fagfeltet kjemikalier og HMS}
\vspace{2pt}
\entry[2013 -- 2019]{\textbf{Guide, Fortidsminneforeningen, sommarjobb}}
\vspace{2pt}

\makerubrichead{Ferdigheiter}

\entry[Kjemiske ferdigheiter]{Erfaring og kompetanse innen organisk kjemi, stråling og strålevern}
\vspace{2pt}
\entry[Datakunnskaper]{Gode datakunnskap og erfaring med materialdataforvaltning}
\vspace{2pt}
\entry[Språk]{sterk munnleg og skriftleg kompetanse i norsk og engelsk, B2-nivå i tysk}
\vspace{2pt}
\entry[Personlige egenskaper]{Strukturert, analytisk og systematisk arbeidsmetodikk, sterkt engasjement, høy grad av integritet og serviceinnstilling, selvgående og mål- og resultatorientert}
\vspace{2pt}

% If you're not a researcher nor an academic, you probably don't have any publications; delete this line.
%% Sometimes when a section can't be nicely modelled with the \entry[]... mechanism; hack our own and use \input NOT \makerubric
% \input{publications}

%\makerubric{misc}

\pagebreak
% \makerubric{referanser}
% \input{referee-full}

\end{document}
